% From IOIC 2021,2022
%我們稱一個二元組 $M = (E, \mathcal{I})$ 為一個擬陣,
%其中 $\mathcal{I} \subseteq 2^E$ 為 $E$ 的子集所形成的\textbf{非空}集合,若:
%
%\begin{itemize}
%    \item 若 $S \in \mathcal{I}$ 以及 $S^\prime \subsetneq S$,則
%        $S^\prime \in \mathcal{I}$
%    \item 對於 $S_1, S_2 \in \mathcal{I}$ 滿足 $|S_1| < |S_2|$,存在
%        $e \in S_2 \setminus S_1$ 使得 $S_1 \cup \{e\} \in \mathcal{I}$
%\end{itemize}
%
%除此之外,我們有以下的定義:
%
%\begin{itemize}
%    \item 位於 $\mathcal{I}$ 中的集合我們稱之為獨立集(independent set),反之不在
%        $\mathcal{I}$ 中的我們稱為相依集(dependent set)
%    \item 極大的獨立集為基底(base)、極小的相依集為迴路(circuit)
%    \item 一個集合 $Y$ 的秩(rank)$r(Y)$ 為該集合中最大的獨立子集,也就是
%        $r(Y) = \max\{|X| \mid X \subseteq Y \text{ 且 } X \in \mathcal{I} \}$
%\end{itemize}
%
%性質:
%\begin{enumerate}
%    \item $X \subseteq Y \land Y \in \mathcal I \implies X \in \mathcal I$
%    \item $X \subseteq Y \land X \notin \mathcal I \implies Y \notin \mathcal I$
%    \item 若 $B$ 與 $B'$ 皆是基底且 $B \subseteq B'$,則 $B=B'$\\
%          若 $C$ 與 $C'$ 皆是迴路且 $C \subseteq C'$,則 $C=C'$
%    \item $e \in E \land X \subseteq E \implies r(X) \leq r(X \cup \{e\}) \leq r(X) + 1$ i.e. 加入一個元素後秩不會降底,最多增加 1
%    \item $\forall Y \subseteq E, \exists X \subseteq Y, r(X)=\lvert X \rvert = r(Y)$
%\end{enumerate}
%
%一些等價的性質:
%\begin{enumerate}
%    \item 對於所有 $X \subseteq E$,$X$ 的極大獨立子集都有相同的大小
%    \item 對於 $B_1,B_2 \in \mathcal B \land B_1 \neq B_2$,
%        對於所有 $e_1 \in B_1 \setminus B_2$,
%        存在 $e_2 \in B_2 \setminus B_1$ 使得 $(B_1 \setminus \{e_1\}) \cup \{e_2\} \in \mathcal B$
%    \item 對於 $X,Y \in \mathcal I$ 且 $\lvert X \rvert < \lvert Y \rvert$,存在 $e \in Y \setminus X$ 使得 $X \cup \{e\} \in \mathcal B$
%    \item 如果 $r(X \cup \{e_1\}) = r(X \cup \{e_2\}) = r(X)$,則 $r(X \cup \{e_1,e_2\}) = r(X)$。
%        如果 $r(X \cup \{e\}) = r(X)$ 對於所有 $e \in E'$ 都成立,則 $r(X \cup E') = r(X)$。
%\end{enumerate}

$M=(E,\mathcal{I})$, where $\mathcal{I} \subseteq 2^E$ is nonempty, is a matroid if:
\begin{itemize}
    \item If $S \in \mathcal{I}$ and $S^\prime \subsetneq S$, then $S^\prime \in \mathcal{I}$.
    \item For $S_1,S_2 \in \mathcal{I}$ s.t. $\lvert S_1 \rvert < \lvert S_2 \rvert$,
        there exists $e \in S_2 \setminus S_1$ s.t. $S_1 \cup \{e\} \in \mathcal{I}$.
\end{itemize}

Matroid intersection:

Start from $S = \emptyset$. In each iteration, let 
%\vspace{-0.5em}
\begin{itemize}
    %\itemsep-0.5em
    \item $Y_1 = \{x \not\in S \mid S \cup \{x\} \in \mathcal{I}_1 \}$
    \item $Y_2 = \{x \not\in S \mid S \cup \{x\} \in \mathcal{I}_2 \}$
\end{itemize}
If there exists $x \in Y_1 \cap Y_2$, insert $x$ into $S$. Otherwise for each $x \in S, y \not\in S$, create edges
%\vspace{-0.5em}
\begin{itemize}
    %\itemsep-0.5em
    \item $x \to y$ if $S - \{x\} \cup \{y\} \in \mathcal{I}_1$.
    \item $y \to x$ if $S - \{x\} \cup \{y\} \in \mathcal{I}_2$.
\end{itemize}
Find a \textit{shortest} path (with BFS) starting from a vertex in $Y_1$ and ending at a vertex in $Y_2$ which doesn't pass through any other vertices in $Y_2$, and alternate the path. The size of $S$ will be incremented by 1 in each iteration. For the weighted case, assign weight $w(x)$ to vertex $x$ if $x \in S$ and $-w(x)$ if $x \not\in S$. Find the path with the minimum number of edges among all minimum length paths and alternate it.

