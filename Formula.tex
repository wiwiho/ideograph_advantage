% modified from kactl
\subsection{Recurrences}
If $a_n = c_1 a_{n-1} + \dots + c_k a_{n-k}$, and $r_1, \dots, r_k$ are distinct roots of $x^k + c_1 x^{k-1} + \dots + c_k$, there are $d_1, \dots, d_k$ s.t.
\[a_n = d_1r_1^n + \dots + d_kr_k^n. \]
Non-distinct roots $r$ become polynomial factors, e.g. $a_n = (d_1n + d_2)r^n$.

\subsection{Geometry}

\subsubsection{Rotation Matrix}

\[ \begin{pmatrix}
    \cos \theta & -\sin \theta \\
    \sin \theta & \cos \theta
\end{pmatrix} \]

\begin{itemize}
    \item rotate $90^\circ$: $(x,y) \to (-y, x)$
    \item rotate $-90^\circ$: $(x,y) \to (y, -x)$
\end{itemize}

\subsubsection{Triangles}
Side lengths: $a,b,c$\\
Semiperimeter: $p=\dfrac{a+b+c}{2}$\\
Area: $A=\sqrt{p(p-a)(p-b)(p-c)}$\\
Circumradius: $R=\dfrac{abc}{4A}$\\
Inradius: $r=\dfrac{A}{p}$\\
Length of median (divides triangle into two equal-area triangles): $m_a=\tfrac{1}{2}\sqrt{2b^2+2c^2-a^2}$\\
Length of bisector (divides angles in two): $s_a=\sqrt{bc\left(1-\left(\dfrac{a}{b+c}\right)^2\right)}$\\
Law of sines: $\dfrac{\sin\alpha}{a}=\dfrac{\sin\beta}{b}=\dfrac{\sin\gamma}{c}=\dfrac{1}{2R}$\\
Law of cosines: $a^2=b^2+c^2-2bc\cos\alpha$\\
Law of tangents: $\dfrac{a+b}{a-b}=\dfrac{\tan\dfrac{\alpha+\beta}{2}}{\tan\dfrac{\alpha-\beta}{2}}$\\
%Incenter:\\
%$P_1=(x_1,y_1),P_2=(x_2,y_2),P_3=(x_3,y_3)$\\
%$s_1=\overline{P_2P_3}, s_2=\overline{P_1P_3}, s_3=\overline{P_1P_2}$ \\
%$\dfrac{s_1P_1 + s_2P_2 + s_3P_3}{s_1+s_2+s_3}$\\
%Circumcenter:\\
%$P_0=(0,0),P_1=(x_1,y_1),P_2=(x_2,y_2)$\\
%$x_c=\frac{1}{2} \times \dfrac{y_2(x_1^2+y_1^2)-y_1(x_2^2+y_2^2)}{-x_2y_1+x_1y_2}$\\
%$y_c=\frac{1}{2} \times \dfrac{x_2(x_1^2+y_1^2)-x_1(x_2^2+y_2^2)}{-x_1y_2+x_2y_1}$\\
%Check if $(x_0,y_0)$ is in the circumcircle:\\
%\[\begin{vmatrix}
%x_1-x_0 & y_1-y_0 & (x_1^2+y_1^2)-(x_0^2+y_0^2) \\
%x_2-x_0 & y_2-y_0 & (x_2^2+y_2^2)-(x_0^2+y_0^2) \\
%x_3-x_0 & y_3-y_0 & (x_3^2+y_3^2)-(x_0^2+y_0^2) \\
%\end{vmatrix}\]
%$0$: on edge, $>0$: inside, $<0$: outside\\

\subsubsection{Quadrilaterals}
With side lengths $a,b,c,d$, diagonals $e, f$, diagonals angle $\theta$, area $A$ and
magic flux $F=b^2+d^2-a^2-c^2$:

\[ 4A = 2ef \cdot \sin\theta = F\tan\theta = \sqrt{4e^2f^2-F^2} \]

 For cyclic quadrilaterals the sum of opposite angles is $180^\circ$,
$ef = ac + bd$, and $A = \sqrt{(p-a)(p-b)(p-c)(p-d)}$.

\subsubsection{Spherical coordinates}
\begin{center}
\includegraphics[width=25mm]{sphericalCoordinates.pdf}
\end{center}
\[\begin{array}{cc}
x = r\sin\theta\cos\phi & r = \sqrt{x^2+y^2+z^2}\\
y = r\sin\theta\sin\phi & \theta = \textrm{acos}(z/\sqrt{x^2+y^2+z^2})\\
z = r\cos\theta & \phi = \textrm{atan2}(y,x)
\end{array}\]

\subsubsection{Green's Theorem}

\[ \iint_D \left( \frac{\partial Q}{\partial x} - \frac{\partial P}{\partial y} \right) dxdy
= \oint_{L^+} (Pdx + Qdy)\]

\[ \text{Area} = \frac{1}{2} \oint_L x\ dy - y\ dx \]

Circular sector:

\begin{align*}
    x &= x_0 + r\cos\theta \\
    y &= y_0 + r\sin\theta \\
    A &= r \int_\alpha^\beta (x_0 + \cos\theta)\cos\theta + (y_0 + \sin\theta)\sin\theta\ d\theta \\
      &= r (r \theta + x_0 \sin\theta - y_0 \cos\theta) \rvert_\alpha^\beta
\end{align*}

\subsubsection{Point-Line Duality}

\[ p=(a,b) \leftrightarrow p^*: y=ax-b \]

\begin{itemize}
    \item $p \in l \iff l^* \in p^*$
    \item $p_1, p_2, p_3$ are collinear $\iff$ $p_1^*, p_2^*, p_3^*$ intersect at a point
    \item $p$ lies above $l$ $\iff$ $l^*$ lies above $p^*$
    \item lower convex hull $\leftrightarrow$ upper envelope
\end{itemize}

\subsection{Trigonometry}
\begin{align*}
    \sinh x = \frac{1}{2}(e^x - e^{-x}) &&& \cosh x = \frac{1}{2}(e^x + e^{-x}) \\
    \sin n\pi = 0 &&& \cos n\pi = (-1)^n
\end{align*}
\begin{align*}
\sin(\alpha+\beta)&{}=\sin \alpha\cos \beta+\cos \alpha\sin \beta\\
\cos(\alpha+\beta)&{}=\cos \alpha\cos \beta-\sin \alpha\sin \beta\\
\sin(2\alpha) &{}=2\cos \alpha \sin \alpha \\
\cos(2\alpha) &{}=\cos^2 \alpha - \sin^2 \alpha \\
         &{}=2\cos^2 \alpha - 1 \\
         &{}=1 - 2 \sin^2 \alpha
\end{align*}
\begin{align*}
\tan(\alpha+\beta)&{}=\dfrac{\tan \alpha+\tan \beta}{1-\tan \alpha\tan \beta}\\
\sin \alpha+\sin \beta&{}=2\sin\dfrac{\alpha+\beta}{2}\cos\dfrac{\alpha-\beta}{2}\\
\cos \alpha+\cos \beta&{}=2\cos\dfrac{\alpha+\beta}{2}\cos\dfrac{\alpha-\beta}{2} \\
\sin \alpha \sin \beta&{}=\frac{1}{2}(\cos(\alpha - \beta) - \cos(\alpha + \beta)) \\
\sin \alpha \cos \beta&{}=\frac{1}{2}(\sin(\alpha + \beta) + \sin(\alpha - \beta)) \\
\cos \alpha \sin \beta&{}=\frac{1}{2}(\sin(\alpha + \beta) - \sin(\alpha - \beta)) \\
\cos \alpha \cos \beta&{}=\frac{1}{2}(\cos(\alpha - \beta) + \cos(\alpha + \beta))
\end{align*}
\[ (V+W)\tan(\alpha-\beta)/2{}=(V-W)\tan(\alpha+\beta)/2 \]
where $V, W$ are lengths of sides opposite angles $\alpha, \beta$.
\begin{align*}
	a\cos x+b\sin x&=r\cos(x-\phi)\\
	a\sin x+b\cos x&=r\sin(x+\phi)
\end{align*}
where $r=\sqrt{a^2+b^2}, \phi=\operatorname{atan2}(b,a)$.

\subsection{Derivatives/Integrals}

Integration by parts:
\[\int_a^bf(x)g(x)dx = [F(x)g(x)]_a^b-\int_a^bF(x)g'(x)dx\]

\begin{align*}
	\dfrac{d}{dx}\arcsin x = \dfrac{1}{\sqrt{1-x^2}} &&& \dfrac{d}{dx}\arccos x = -\dfrac{1}{\sqrt{1-x^2}} \\
	\dfrac{d}{dx}\tan x = 1+\tan^2 x &&& \dfrac{d}{dx}\arctan x = \dfrac{1}{1+x^2} \\
	\int\tan ax = -\dfrac{\ln|\cos ax|}{a} &&& \int x\sin ax = \dfrac{\sin ax-ax \cos ax}{a^2} \\
	\int e^{-x^2} = \frac{\sqrt \pi}{2} \text{erf}(x) &&& \int xe^{ax} = \frac{e^{ax}}{a^2}(ax-1) \\
    \int \sin^2(x) = \frac{x}{2} - \frac{1}{4} \sin 2x &&& \int \sin^3 x = \frac{1}{12}\cos 3x - \frac{3}{4} \cos x \\
    \int \cos^2(x) = \frac{x}{2} + \frac{1}{4} \sin 2x &&& \int \cos^3 x = \frac{1}{12}\sin 3x + \frac{3}{4} \sin x \\
    \int x \sin x = \sin x - x \cos x &&& \int x \cos x = \cos x + x \sin x \\
    \int xe^x = e^x(x - 1) &&& \int x^2 e^x = e^x(x^2 - 2x + 2) \\
\end{align*}
\begin{align*}
    \int x^2 \sin x &= 2x \sin x - (x^2 - 2) \cos x \\
    \int x^2 \cos x &= 2x \cos x + (x^2 - 2) \sin x \\
    \int e^x \sin x &= \frac{1}{2}e^x (\sin x - \cos x) \\
    \int e^x \cos x &= \frac{1}{2}e^x (\sin x + \cos x) \\
    \int xe^x \sin x &= \frac{1}{2}e^x (x \sin x - x \cos x + \cos x) \\
    \int xe^x \cos x &= \frac{1}{2}e^x (x \sin x + x \cos x - \sin x)
\end{align*}

\subsection{Sums}
\[ c^a + c^{a+1} + \dots + c^{b} = \frac{c^{b+1} - c^a}{c-1}, c \neq 1 \]
\begin{align*}
	1 + 2 + 3 + \dots + n &= \frac{n(n+1)}{2} \\
	1^2 + 2^2 + 3^2 + \dots + n^2 &= \frac{n(2n+1)(n+1)}{6} \\
	1^3 + 2^3 + 3^3 + \dots + n^3 &= \frac{n^2(n+1)^2}{4} \\
	1^4 + 2^4 + 3^4 + \dots + n^4 &= \frac{n(n+1)(2n+1)(3n^2 + 3n - 1)}{30} \\
\end{align*}

\subsection{Series}
$$e^x = 1+x+\frac{x^2}{2!}+\frac{x^3}{3!}+\dots,\,(-\infty<x<\infty)$$
$$\ln(1+x) = x-\frac{x^2}{2}+\frac{x^3}{3}-\frac{x^4}{4}+\dots,\,(-1<x\leq1)$$
$$\sqrt{1+x} = 1+\frac{x}{2}-\frac{x^2}{8}+\frac{2x^3}{32}-\frac{5x^4}{128}+\dots,\,(-1\leq x\leq1)$$
$$\sin x = x-\frac{x^3}{3!}+\frac{x^5}{5!}-\frac{x^7}{7!}+\dots,\,(-\infty<x<\infty)$$
$$\cos x = 1-\frac{x^2}{2!}+\frac{x^4}{4!}-\frac{x^6}{6!}+\dots,\,(-\infty<x<\infty)$$

